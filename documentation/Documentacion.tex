\documentclass[12pt,a4paper]{article}
\usepackage[utf8]{inputenc}
\usepackage[spanish]{babel}
\usepackage{amsmath}
\usepackage{amsfonts}
\usepackage{amssymb}
\usepackage{makeidx}
\usepackage{graphicx}
\usepackage[left=2cm,right=2cm,top=2cm,bottom=2cm]{geometry}
\author{Edgar Luque}
\title{Proyecto Final}
\begin{document}

\begin{titlepage}
    \begin{center}
        \vspace*{1cm}
            
        \Huge
        \textbf{Gymodo}
            
        \vspace{0.5cm}
        \LARGE
        La mejor App para tu gym.
        
        \vfill
        

        Edgar Luque, Shah Sawar, Ronald Intriago\\
            
        \vspace{0.8cm}
           
            
        \Large
        Desarrollo de Aplicaciones Multiplataforma\\
        Escola del Treball\\
        Barcelona\\
        \today
            
    \end{center}
\end{titlepage}

\newpage

\begin{abstract}
Gyomodo es una aplicación que tiene como objetivo resolver los problemas que puedan tener los gymnasios en estos tiempos modernos, pero sobretodo, problemas originados a partir de la pandémia del Covid-19.

Este documento explica el desarrollo de esta aplicación, su funcionalidad y la organización del equipo.
\end{abstract}

\newpage

\tableofcontents

\newpage

\section{Analísis funcional}

\subsection{Diagrama UML}
UML aqui

\subsection{Diagrama Casos de Uso}
Casos de uso aqui

\subsection{Diagrama Relacional (bases de datos)}
Diagrama aqui

\subsection{Mockups}
Imagenes de mockups aqui

\subsection{Tecnología usada}
Git, asana, etc

\end{document}