\documentclass[12pt,a4paper]{article}
\usepackage[utf8]{inputenc}
\usepackage[spanish]{babel}
\usepackage{amsmath}
\usepackage{amsfonts}
\usepackage{amssymb}
\usepackage{makeidx}
\usepackage{hyperref}
\usepackage{graphicx}
\usepackage{fancyhdr}
\usepackage{wrapfig}
\usepackage{float}
\usepackage{placeins}
\usepackage{afterpage}
\usepackage[export]{adjustbox}
\usepackage{multicol}
\usepackage[left=2cm,right=2cm,top=2cm,bottom=2cm]{geometry}
\graphicspath{ {./images/} }

\title{Gymodo}

\pagestyle{fancy}
\fancyhf{}
\rhead{Gymodo}
\lhead{M13 Proyecto Desarrollo de aplicaciones multiplataforma}
\fancyfoot[C,CO]{\leftmark}
\fancyfoot[L,RO]{\thepage}

\renewcommand{\headrulewidth}{2pt}
\renewcommand{\footrulewidth}{1pt}


\begin{document}

\begin{titlepage}
    \begin{center}
        \vspace*{1cm}
            
        \Huge
        \textbf{Gymodo}
            
        \vspace{0.5cm}
        \LARGE
        La mejor App para tu gym
        
        \includegraphics[width=\textwidth]{gymodo_logo}
        
        \vfill
        

        Edgar Luque, Shah Sawar, Ronald Intriago\\
            
        \vspace{0.8cm}
           
            
        \Large
        Desarrollo de Aplicaciones Multiplataforma\\
        Escola del Treball\\
        Barcelona\\
        \today
            
    \end{center}
\end{titlepage}

\newpage

\begin{abstract}
Gyomodo es una aplicación que tiene como objetivo resolver los problemas que puedan tener los gymnasios en estos tiempos modernos, pero sobretodo, problemas originados a partir de la pandemia del Covid-19.

Este documento explica el desarrollo de esta aplicación, su funcionalidad y la organización del equipo.
\end{abstract}

\newpage

\tableofcontents

\newpage

\section{Presentación del proyecto}
Este proyecto se basa en el desarrollo de una aplicación para Android, esta aplicación tiene como objetivo principal cubrir las necesidades digitales que puede tener un gimnasio como:

\begin{itemize}
\item Crear rutinas y ejercicios.
\item Reservar una hora para ir al gimnasio.
\item Crear tus propias dietas y escanear el código de barras de los productos para ver su nutrientes.
\item Ver noticias relacionadas con el mundo del ejercicio.
\end{itemize}

La función que creemos mas importante es la de reservar hora en un gimnasio, ya que en estos tiempos de pandemia se puede requerir pedir hora previa antes de ir a un gimnasio.

Usando la librería \textbf{ML Kit} de Google podemos escanear el código de barras que tienen los productos, esto nos da una id que podemos usar para buscar el producto usando la API del servicio \textbf{Open Food Facts} que es una base de datos abierta sin animo de lucro con información sobre productos alimenticios "hecha por todos y para todos".

\newpage

\section{Estructura y Organización}
TODO EXPLICAR ORGANIZACION AQUI

\newpage

\section{Tecnología usada}

\subsection{Organización}

\subsubsection{Asana}


\begin{minipage}{.75\textwidth}
Para organizar las tareas que tenemos que hacer hemos empleado Asana, una aplicación web que permite organizar el trabajo.


Con Asana podemos crear tareas, sub-tareas y asignarlas a cada uno, también permite poner un tiempo limite. 
\end{minipage} %
\begin{minipage}{.25\textwidth}
  \includegraphics[width=0.7\textwidth, right]{asana}
\end{minipage}


\subsubsection{Toggl}

\begin{minipage}{.75\textwidth}
Para saber el tiempo empleado en cada tarea usamos la herramienta toggl tracker.
\end{minipage} %
\begin{minipage}{.25\textwidth}
  \includegraphics[width=0.7\textwidth, right]{toggl}
\end{minipage}

\subsection{Desarrollo}

\subsubsection{Control de Versiones}

\begin{minipage}{.75\textwidth}
El sistema de control de versiones que hemos usado es git, gracias a esta herramienta podemos mantener el proyecto de forma eficiente, este es nuestra forma de trabajar:

\begin{enumerate}
\item Actualizar la branca rama principal.
\item Crear una rama donde guardaras tu nuevo trabajo.
\item Hacer el trabajo y subirlo.
\item Otro miembro revisa el código y si está bien se hace un merge a la rama principal.
\item Repetir el paso 1.
\end{enumerate}
\end{minipage} %
\begin{minipage}{.25\textwidth}
  \includegraphics[width=0.7\textwidth, right]{git}
\end{minipage}


\subsubsection{Android Studio}

\begin{minipage}{.75\textwidth}
Para desarrollar la aplicación hemos usado el IDE Android Studio.
Este IDE es el estándar de la industria para crear aplicaciones de Android, está desarrollada por Google y Jetbrains.
\end{minipage} %
\begin{minipage}{.25\textwidth}
  \includegraphics[width=0.7\textwidth, right]{androidstudio}
\end{minipage}


\subsubsection{Firebase}

\begin{minipage}{.75\textwidth}
Es una plataforma que te permite crear aplicaciones web y para el móvil, este nos proporciona una base de datos en la nube y una API para interactuar con esta. También nos permite guardar imágenes, ver estadísticas sobre los usuarios que usan la app y más.
\end{minipage} %
\begin{minipage}{.25\textwidth}
  \includegraphics[width=0.7\textwidth, right]{firebase}
\end{minipage}

\subsubsection{ML Kit}

\begin{minipage}{.75\textwidth}
Es una librería de Google que usa \textit{Machine Learning} y permite realizar acciones como detección de cara, escanear barcodes, identificar objectos, texto, etc.
\end{minipage} %
\begin{minipage}{.25\textwidth}
  \includegraphics[width=0.7\textwidth, right]{mlkit}
 \end{minipage}

\newpage

\section{Diseño}
El objetivo del diseño es dar al usuario una experiencia fluida y cómoda mientras hace uso de la aplicación, por lo que cada elemento se sitúa en lugares de la pantalla que el usuario encontrará intuitivo.
Además, se han hecho uso del contraste entre colores y varias leyes de Gestalt, tales como la ley de la simetría y la experiencia.

\subsection{Logo}
El logo se compone de un símbolo y texto; el símbolo es un brazo con cierta musculatura levantando una pesa, en este símbolo se puede advertir la forma de un corazón, dando a entender el amor por el \textit{fitness}.
El texto es el nombre de la aplicación usando nuestro color principal y la fuente \textbf{Montserrat} en negrita, el resto de la aplicación también hace uso de ésta.


\subsection{Colores}

Los efectos del ejercicio en el ser humano son positivos; reduce el estrés, la ansiedad y la depresión gracias a la segregación de \textbf{endorfinas}, unas sustancias químicas que producen sensación de felicidad y euforia.

Hemos querido representar esos efectos en nuestro diseño a través del color \textbf{naranja} como color principal, ya que, según la psicología del color, el naranja es energético y se asocia con la juventud, alegría y dinamismo.
Para destacar nuestro color principal y hacer contraste, se han escogido colores oscuros y el blanco. Estos colores oscuros se asocian con la fortaleza; el blanco, con la paz y tranquilidad, estos efectos se consiguen también al tener una vida con una actividad física presente.

\begin{figure}[h]
  \centering
 \includegraphics[width=1\textwidth]{color_palette}
 \caption{Paleta de colores}
\end{figure}

\section{Documentación Técnica}

\subsection{Navigation Component}

\begin{figure}[h]
  \centering
 \includegraphics[width=1\textwidth]{navComExample}
 \caption{Nav Component}
\end{figure}

\subsection{Diagrama UML}

\begin{figure}[h]
 	\centering
	\includegraphics[width=0.6\textwidth]{uml}
	\caption{Diagrama UML}
\end{figure}

\newpage

\subsection{Diagrama Casos de Uso}

\begin{figure}[h]
 	\centering
	\includegraphics[width=0.6\textwidth]{casos_uso}
	\caption{Diagrama de Casos de Uso}
\end{figure}

TODO: Explicar casos de uso paso a paso. e.g: 1. seleccionar boton x, hacer x, etc

\newpage

\subsection{Diagrama Relacional (bases de datos)}

\begin{figure}[h]
 	\centering
	\includegraphics[width=0.6\textwidth]{diagramaer}
	\caption{Diagrama Entidad Relacion}
\end{figure}

\newpage

\subsection{Mockup}
\href{https://mockittapp.wondershare.com/app/3398ef738f7ac4f35fae5df4eb77004473612d19?simulator_type=device&sticky}{Link al mockup}

todo: poner foto

\newpage

\section{Estadísticas sobre el proyecto}

\subsection{Contribuciones}

\begin{figure}[h]
 	\centering
	\includegraphics[width=\textwidth]{git-contributors}
	\caption{edg-l = Edgar, JRIP24 = Ronald, shasawar = Shah}
\end{figure}

\subsection{Lineas de código}

\begin{verbatim}
===============================================================================
 Language            Files        Lines         Code     Comments       Blanks
===============================================================================
 Batch                   1           84           61            0           23
 Java                   69         8648         5670         1453         1525
 JSON                    1           47           47            0            0
 Markdown                1            2            0            2            0
 Prolog                  1           21           18            0            3
 Shell                   1          172          130           23           19
 TeX                     1          200          136            0           64
 XML                    96         4280         3760          100          420
===============================================================================
 Total                 171        13454         9822         1578         2054
===============================================================================
\end{verbatim}

\newpage

\section{Conclusión}

\subsection{Posibles ampliaciones}

\subsubsection{Ampliar los datos de la comida}
Se podrían mostrar mas datos sobre la comida que se escanea, y a partir de estos datos llegar a conclusiones útiles para el usuario.

\end{document}